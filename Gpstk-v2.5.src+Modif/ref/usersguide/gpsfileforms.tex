\chapter{GPS File Formats}

A variety of file formats are supported within the GPSTk.  
The file formats generally store GPS observation data or data related 
to processing of GPS observables.  In this section, a summary of the 
file formats supported within the GPSTk is presented along with a brief 
rationale of why each format is supported within the GPSTk and where to
find additional information on the format.  

\section{RINEX}
The Receiver INdependent EXchange (RINEX) format was developed by the
National Geodetic Survey (NGS) in the U.S. and the University of Berne 
in Switzerland.  RINEX is actually three format definitions that allow 
storage of GPS observations, GPS navigation message information, 
and meteorological data associated with GPS observations.  
GPSTk contains classes to both read and write RINEX V2.1 and V3 data files 
of all types (observation, navigation message, and meteorological). 
RINEX has undergone a number of revisions since its inception. Each
revision is defined using a standard \cite{rinex1format}, \cite{rinex2format},
\cite{rinex211format}, \cite{rinex300format}.

\section{FIC}
The Floating, Integer, Character (FIC) format was developed in the mid-80�s as
a relatively machine-independent way to store GPS observation and navigation
message data while retaining receiver specific characteristics.  
Over time, the RINEX format (see above) proved more popular with users
and use of the observation records within the FIC format faded away.  
However, the FIC records associated with GPS navigation message data are
still supported within the GPSTk because these records retain some data 
quantities that are not contained within the RINEX navigation message file.
For example, RINEX makes few provisions for storing the almanac data
contained in Subframe 4 and Subframe 5. Like RINEX, a standards document
defines FIC \cite{ficproposal}.

\section{SP-3}
The SP-3 format stores ephemeris information for satellites.
Usually SP-3 is used for storage of GPS precise ephemerides.
GPSTk supports both SP-3a and SP3-c formats.  SP-3 was originally designed
by NGS.  Standards documents describe the specific details of the SP-3 formats
\cite{sp3format:ngs}, \cite{sp3format:igscb}.

\putbib[gpstk]
