\chapter{Using rtAshtech}

By R. Benjamin Harris

\section{Overview}

\texttt{rtAshtech} is an application based on the GPSTk [1] that logs 
observations from the Ashtech Z-XII receiver. It records observations and 
navigation messages in the RINEX 2.1 format. [2] It can also 
generate observation files in a space delimited formated 
that can be directly loaded into numerical programs
such as Excel, MATLAB and Octave. The raw messages from the receiver 
interface can optionally be captured. Finally, \texttt{rtAshtech} 
can produce a log file to give the user insight into its operation.


\section{Background}

Modern, survey-grade GNSS receiverscan store observations internally
for a number of days. Then those observations can be extracted from the
receiver using software provided by the receiver manufacturer or using
open source tools[3]

Sometimes it is necessary to collect more observations for a period longer
than a receiver can store. In those cases, a data logging program must be
employed. Most receiver manufacturers provide such a program for their 
receivers. However, in the case of older such as the Ashtech Z-XII,
those programs are no longer supported. Furthermore, those programs
cannot function under Linux. It is possible to use the program 
\texttt{teqc} to convert
Ashtech observations in real time, however the output stream must
be managed by the user. 


\subsection{The Ashtech Z-XII Receiver}

The Z-XII is a survey grade receiver introduced by Ashtech in the early 1990's.
It is able to track both frequencies of GPS using what is referred to as a 
``codeless'' tracking scheme. That is, the Z-XII track the military signals on
L2 based on correlation of the received signal with P-code, and without 
specific knowledge of Y-code. This mode of tracking is fully documented in
a series of U.S. patents.[4]

\subsection{Ashtech Receiver Message Structure}

The Ashtech Z-XII communicates over a serial port using a proprietary
format. That format is defined partially in the Z-XII reference manual. [5]
Some details, such as the interpretation of signal-to-noise messages--have been
shared only in memos. [6] Finally, Ashtech has documented the interface
for not only the Ashtech but also a related family of receivers in a 2002
publication [7] 


\section{Detailed Description}

\subsection{Setting up the Receiver}

In the current version of \texttt{rtAshtech}, the receiver must be configured 
by hand to transmit observations and ephemerides. Specifically the following 
messages must be enabled from the receiver front panel.
 
\begin{center}
\begin{tabular}{|l|l|}
\hline
\multicolumn{2}{|c|}{Manually Enabled Messages} \\
\hline
ASCII PBEN   & Text formatted position and clock solutions \\
ASCII MBEN   & Text formatted observations \\
EPB          & Binary ephmerides \\
\hline
\end{tabular}
\end{center}

Note that the ASCII, not BINARY, encoding is required for this version of
\texttt{rtAshtech}. 

The user will need to manually set other interface options. The data rate 
must be set. The communication speed must be set to 115200 baud. Because
the data rate is not checked or enforced by \texttt{rtAshtech}, then
it is possible for the user to change the data rate while logging 
data without restarting the program.

\subsection{Setting up the Software}

The serial port on the data collection computer does not need to be 
configured. However it is suggested that a terminal program like
\texttt{minicom} be used to verify that the serial communication is 
receiving data.

The contents of the headers is under the direct control of the user. 
Header templates files are read when \texttt{rtAshtech} starts. If additional
comments or site information should be added to the header, then the
templates can be edited with a basic text editor. Note that it not
necessary to correct header entries associated with timestamps, e.g., 
``TIME OF FIRST OBS'' . Because \texttt{rtAshtech} must read these
files on startup, those files must be in the current working directory
when the application is started.


\section{Execution}

The user executes \texttt{rtAshtech} from the command line. The processing
performed by the \texttt{rtAshtech} is specified through command line
arguments. The full set of arguments is defined below. This list can be
duplicated by running \texttt{rtAshtech -h} on the command line.

\small
\begin{singlespace}
\begin{verbatim}
user@host:~$ /rtAshtech -h
Usage: rtAshtech [OPTION] ...
Records observations from an Ashtech Z-XII receiver.

Optional arguments:
  -h, --help          Print help usage
  -v, --verbose       Increased diagnostic messages
  -r, --raw           Record raw observations
  -l, --log           Record log entries
  -t, --text          Record observations as simple text files
  -p, --port=ARG      Serial port to use
  -o, --rinex-obs=ARG Naming convention for RINEX obs files
  -n, --rinex-nav=ARG Naming convention for RINEX nav message files
  -T, --text-obs=ARG  Naming convention for obs in simple text files
\end{verbatim}
\end{singlespace}
\normalsize

The serial port argument is a text string describing the full
path to the serial device. For example, in Linux, the argument could 
be \texttt{-p /dev/ttyS0} to connect to the first serial port.

The options that describe the file naming convention all work similarly.
The file naming convention is precisely the time specification used 
throughout the GPSTk. For example, ``surv\%03d.\%02yo'' creates file with names that start with
``surv,'' followed by a three digit day of the year, then a period, then a
two digit year number, then by the character ``o.''  Every item to be 
recorded--either observation or navigation message--
is mapped to a file name based on the associated epoch.
If that file does not exist, a new output file is created. 

Files are written to the current working directory by convention. The
user can specify another output directory by embedding that directory
into the filenaming convention. For example, to output RINEX observation
files to the directory \texttt{/opt/website/data/rinex/obs}, then the
file specification could be 
\texttt{-o ``/opt/website/data/rinex/obs/surv\%03d.\%02yo''}.

By altering the filenaming convention, the user effectively controls the time
span of data contained in the output files. For example, the filenaming 
convention ``obs\%G.txt'' could be used to create observation files that 
contain a full weeks worth of observations.

Note that the observations can be written into a simple text file. Such
files can be directly loaded into numerical software such as Excel, MATLAB
or Octave. 
\subsection{Display}

The output of \texttt{rtAshtech} is text based. Three kinds of information
are presented to the user. First, the tracking status printed by channel.
Second, the status of communications buffers are listed. Then, log
messages are printed. If the \texttt{-l} log option is used, then the
log messages are captured to a file. 

Here is an example output

\small
\begin{singlespace}
\begin{verbatim}
GPSTk Real-Time Data Collection for the Ashtech Z-XII ver. 1.0

Channel    1    2    3    4    5    6    7    8    9   10   11   12
PRN        5   22   25   18   30    9   21   15   19    3   14    1

Number of unprocessed characters in buffer: 358
Today's message count:                      49
Number of unproceseed obs:                  11

Log Messages
-----------------------------------------------------------------------------
 05/04/2006 20:59:10.6 - Got ephemeris for PRN 3
 05/04/2006 20:59:10.6 - Converted a nav message
 05/04/2006 20:59:10.6 - Wrote nav message
 05/04/2006 20:59:10.6 - Got ephemeris for PRN 14
 05/04/2006 20:59:10.6 - Converted a nav message
 05/04/2006 20:59:10.6 - Wrote nav message
 05/04/2006 20:59:11.2 - Got ephemeris for PRN 1
 05/04/2006 20:59:11.2 - Converted a nav message
 05/04/2006 20:59:11.2 - Wrote nav message
 05/04/2006 20:59:11.2 - Opened output file: site124.06o
-----------------------------------------------------------------------------
\end{verbatim}
\end{singlespace}
\normalsize

The time associated with the log files is the UTC time realized by the host
machine.

Note that more information is output if the \texttt{-v} verbose option
is used. In fact, more information is printed than can be contained in 
the default 80x25 consoles used by most Linux-based machines. However this
is not a problem for the resizeable pseudo-terminals provided by most
X windows environments.

\section{Examples and Usage Notes}

This section contains a number of practical examples in the use of
\texttt{rtAshtech}. In each subsection there is a brief description of the
resulting processing.

\subsection{Basic recording of RINEX files}

By default, \texttt{rtAshtech} records just RINEX obs and nav files.
The following example demonstrates this
most basic processing mode of \texttt{rtAshtech}.

\small
\begin{singlespace}
\begin{verbatim}
user@host:~$ ./rtAshtech
\end{verbatim}
\end{singlespace}
\normalsize

In this case the output is written to the current terminal. For long
term recording, the user may want to enable rtAshtech to continue
running after the user has logged out. In Linux the following command
meets that need.

\small
\begin{singlespace}
\begin{verbatim}
user@host:~$ nohup ./rtAshtech > /dev/null &
\end{verbatim}
\end{singlespace}
\normalsize


\subsection{Recording RINEX using a non-Default Serial Port}

By default, \texttt{rtAshtech} connect to the first serial port. It is
possible to specify another serial port with the \texttt{-p} option. 
The following example demonstrates how to the second serial port.

\small
\begin{singlespace}
\begin{verbatim}
user@host:~$ ./rtAshtech -p /dev/ttyS1
\end{verbatim}
\end{singlespace}
\normalsize

\subsection{Recording Files to User-Defined Locations}

Recall that the output files are defined using a user-defined specification.
That specification uses the naming convention used through all GPSTk
applications. If the input data maps to a new filename, then the current
file is closed and the new file is opened. If a new obs file is desired 
every minute, then the user could start the application as follows:

\small
\begin{singlespace}
\begin{verbatim}
./rtAshtech -o "minute%03j%02H%02M.%02yo"
\end{verbatim}
\end{singlespace}
\normalsize

If the time when the application was start was May 4, 2006, at 20:16 UTC, then
a series of observation files would be created: minute1242016.06o, 
minute1242017.06o, minute1242018.06o, etc. 

\section{References}

\begin{enumerate}

  \item Brian Tolman, R. Benjamin Harris, Tom Gaussiran, David Munton, Jon Little, Richard Mach, Scot Nelsen, Brent Renfro, ARL:UT; David Schlossberg, University of California Berkeley. ``The GPS Toolkit -- Open Source GPS Software.'' Proceedings of the 16th International Technical Meeting of the Satellite Division of the Institute of Navigation (ION GNSS 2004). Long Beach, California. September 2002.

  \item RINEX: The Receiver Independent Exchange Format Version 2.10.
  Available on the web at http://www.ngs.noaa.gov/CORS/Rinex2.html.

  \item The Sharc project. http://sharc.sourceforge.net/

  \item United States Patent and Trademark Office: http://www.uspto.gov/, Patents 5293170, 5134407, and 4928106, owned by Ashtech Telesis, Inc.

  \item Ashtech XII GPS Receiver Operating Manual.
  
  \item Various memos from Ashtech to ARL:UT.

  \item Z-Family GPS Receivers Technical Reference Manual. Thales Navigation. 2002.
  
\end{enumerate}

