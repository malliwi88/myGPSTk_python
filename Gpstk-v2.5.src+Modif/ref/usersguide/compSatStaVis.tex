%\documentclass{article}
%\usepackage{fancyvrb}
%\usepackage{perltex}
%\usepackage{xcolor}
%\usepackage{listings}
%\usepackage{longtable}
%\usepackage{multirow}
%\RecustomVerbatimEnvironment{Verbatim}{Verbatim}{frame=single}
\definecolor{console}{rgb}{0.95,0.95,0.95}
\lstset{basicstyle=\ttfamily, columns=flexible, backgroundcolor=\color{console}}

\newcommand{\outputsize}{footnotesize}
\newcommand{\application}[1]{\emph{#1}}
\newcommand{\setconsole}{\lstset{basicstyle=\ttfamily, columns=flexible, backgroundcolor=\color{console}}}
\newcommand{\setfileio}{\lstset{basicstyle=\ttfamily, columns=flexible, backgroundcolor=\color{console}}}

\perlnewcommand{\getuse}[1]
{
        my $command = $_[0];
        $command = $command." > temp 2>&1 |head -n 15";
        system("$command");

	my $counter = 0;
	my $done = 0;
	my $output = "";
        open(input,"temp");

        while(my $line = <input>)
	{
		if($done == 0)
		{
			$output = $output.$line if $counter < 15;
			$output = $output." . . .\n" if $counter >= 15;
			$done = 1 if $counter >= 15;
		}

		$counter = $counter + 1;
	}

        close(input);

        return  "\\begin{\\outputsize}\n" . "\\begin{lstlisting}\n" .
                "> ".$_[0]."\n\n" . $output .
                "\\end{lstlisting}\n" . "\\end{\\outputsize}\n";
}

\perlnewcommand{\getrevision}[1]
{
	my $revision = $_[0];
	$revision =~ m/\$LastChangedRevision: ([^\$]*)/;
	return $1;
}

\perlnewcommand{\entry}[4]
{
	my $output = "$_[0] \& $_[1] \& \\multirow{$_[3]}{2.5in}{$_[2]} \\\\";

	my $cnt = 0;
	while($cnt < ($_[3]-1))
	{
		$output = $output." \& \& \\\\ ";
		$cnt = $cnt + 1;
	}
	
	return $output;
}


%\begin{document}

\index{compSatVis!application writeup}
\index{compStaVis!application writeup}

\section{\emph{compSatVis compStaVis}}
\subsection{Overview}
\application{compSatVis} computes satellite visibility. \application{compStaVis} computes station visibility.


\subsection{Usage}
\subsubsection{\emph{compSatVis compStaVis}}

\begin{\outputsize}
\begin{longtable}{lll}
%\multicolumn{3}{c}{\application{compSatVis} \application{compStaVis}} \\
\multicolumn{3}{l}{\textbf{Required Arguments}} \\
\entry{Short Arg.}{Long Arg.}{Description}{1}
\entry{-o}{--output-file=ARG}{Name of the output file to write.}{1}
\entry{-n}{--nav=ARG}{Name of navigation file.}{1}
\entry{-c}{--mscfile=ARG}{Name of MS coordinates file.}{1}
& & \\
\multicolumn{3}{l}{\textbf{Optional Arguments}} \\
\entry{Short Arg.}{Long Arg.}{Description}{1}
\entry{-d}{--debug}{Increase debug level.}{1}
\entry{-v}{--verbose}{Increase verbosity.}{1}
\entry{-h}{--help}{Print help usage.}{1}
\entry{-p}{--int=ARG}{Interval in seconds.}{1}
\entry{-e}{--minelv=ARG}{Minimum elevation angle.}{1}
\entry{-t}{--navFileType=ARG}{FALM (FIC Almanac), FEPH (FIC Ephemeris), RNAV, YUMA, SEM (System Effectiveness Model), or SP3.}{3}
\entry{-m}{--min-sta=ARG}{Minimum number of stations visible simultaneously. compStaVis only.}{2}
\entry{-m}{--max-SV=ARG}{Maximum number of SVs tracked simultaneously.  compSatVis only.}{2}
\entry{-D}{--detail}{Print SV count for each interval.}{1}
\entry{-x}{--exclude=ARG}{Exclude station.}{1}
\entry{-i}{--include=ARG}{Include station.}{1}
\entry{-s}{--start-time=TIME}{Start time of evaluation ("m/d/y H:M").}{1}
\entry{-z}{--end-time=TIME}{End time of evaluation ("m/d/y H:M").}{1}
\end{longtable}
\end{\outputsize}

\subsection{Examples}

\subsubsection{Generating satellite visibility statistics using the SEM almanac from the USCG Navigation Center.}
This example loads SEM almanac data from the file current.al3 and a list of station locations from the file stations.msc. It then calculates the number of satellites visible to each station found at each 60 sec interval from 0000Z to 2356Z of Jan 13, 2008. using a 10 degree minimum elevation angle. The results are written to the file visout.txt. Note the use of a specific start time. The SEM and Yuma almanac formats contain an almanac reference week, which is generally in the range 0-1023 (the existing format definitions are ambiguous and SEM and Yuma almanacs with full week numbers have been reported, at least anecdotally). If the -s command is not specified, compSatVis will use whatever reference time is given in the almanac file, which may result in unexpected results.
\begin{verbatim}
user@host:~$ compSatVis -ovisout.txt -ncurrent.al3 -tSEM 
-cstations.msc -e10 -p60 -s"01/16/2008 00:00"
\end{verbatim}

\subsubsection{Generating station visibility statistics using the SEM almanac from the USCG Navigation Center.}
Same as the previous example, however, the values calculated and the statistics will reflect the number of stations visible to each satellite.
\begin{verbatim}
user@host:~$ compSatVis -ovisout.txt -ncurrent.alm -tYUMA 
-cstations.msc -e10 -p60 -s"01/13/2008 00:00" -z"01/16/2008 23:59"
\end{verbatim}

\subsubsection{Generating satellite visibility statistics using the Yuma almanac from the USCG Navigation Center.}
Similar to the first example, but the statistics are computed over four complete days.
\begin{verbatim}
user@host:~$ compSatVis -ovisout.txt -ncurrent.alm -tYUMA 
-cstations.msc -e10 -p60 -s"01/13/2008 00:00" -z"01/16/2008 23:59"
\end{verbatim}

\subsubsection{Generating satellite visibility statistics using SP3 files. }
Similar to the first example, however, navigation message data are from three SP3 files. It is necessary to load three SP3 files to cover the default sidereal day period because the methods that calculate SV positions from the SP3 data use interpolation and need data from the previous day and the following day in order to have sufficient points for the interpolation. In this example in which no evaluation period is specified, compSatVis derives coverage for the "middle day" for the period. 
\begin{verbatim}
user@host:~$ compSatVis -ovisout.txt -napc14622 -napc14623 -napc14624 
-tSP3 -cstations.msc -e10 -p60
\end{verbatim}

%\end{document} 
