%\documentclass{article}
%\usepackage{fancyvrb}
%\usepackage{perltex}
%\usepackage{xcolor}
%\usepackage{listings}
%\usepackage{longtable}
%\usepackage{multirow}
%\RecustomVerbatimEnvironment{Verbatim}{Verbatim}{frame=single}
\definecolor{console}{rgb}{0.95,0.95,0.95}
\lstset{basicstyle=\ttfamily, columns=flexible, backgroundcolor=\color{console}}

\newcommand{\outputsize}{footnotesize}
\newcommand{\application}[1]{\emph{#1}}
\newcommand{\setconsole}{\lstset{basicstyle=\ttfamily, columns=flexible, backgroundcolor=\color{console}}}
\newcommand{\setfileio}{\lstset{basicstyle=\ttfamily, columns=flexible, backgroundcolor=\color{console}}}

\perlnewcommand{\getuse}[1]
{
        my $command = $_[0];
        $command = $command." > temp 2>&1 |head -n 15";
        system("$command");

	my $counter = 0;
	my $done = 0;
	my $output = "";
        open(input,"temp");

        while(my $line = <input>)
	{
		if($done == 0)
		{
			$output = $output.$line if $counter < 15;
			$output = $output." . . .\n" if $counter >= 15;
			$done = 1 if $counter >= 15;
		}

		$counter = $counter + 1;
	}

        close(input);

        return  "\\begin{\\outputsize}\n" . "\\begin{lstlisting}\n" .
                "> ".$_[0]."\n\n" . $output .
                "\\end{lstlisting}\n" . "\\end{\\outputsize}\n";
}

\perlnewcommand{\getrevision}[1]
{
	my $revision = $_[0];
	$revision =~ m/\$LastChangedRevision: ([^\$]*)/;
	return $1;
}

\perlnewcommand{\entry}[4]
{
	my $output = "$_[0] \& $_[1] \& \\multirow{$_[3]}{2.5in}{$_[2]} \\\\";

	my $cnt = 0;
	while($cnt < ($_[3]-1))
	{
		$output = $output." \& \& \\\\ ";
		$cnt = $cnt + 1;
	}
	
	return $output;
}


%\begin{document}

\index{ephdiff!application writeup}
\section{\emph{ephdiff}}
\subsection{Overview}
The application compares the contents of two files containing ephemeris data.

\subsection{Usage}
\subsubsection{\emph{ephdiff}}

\begin{\outputsize}

\begin{longtable}{lll}
\%multicolumn{3}{c}{\application{ephdiff}} \\
\multicolumn{3}{l}{\textbf{Optional Arguments}} \\
\entry{Short Arg.}{Long Arg.}{Description}{1}
\entry{-d}{--debug}{Increase debug level.}{1}
\entry{-v}{--verbose}{Increase verbosity.}{1}
\entry{-h}{--help}{Print help usage.}{1}
\entry{-f}{--fic=ARG}{Name of an input FIC file.}{1}
\entry{-r}{--rinex=ARG}{Name of an input RINEX NAV file.}{1}

\end{longtable}

\end{\outputsize}

\subsection{Examples}
\begin{\outputsize}
\begin{verbatim}
> ephdiff -f fic06.187 -r arl2800.06n

Broadcast Ephemeris (Engineering Units)

PRN : 11

              Week(10bt)     SOW     DOW   UTD     SOD   MM/DD/YYYY   HH:MM:SS
Clock Epoch:  1382( 358)  417600   Thu-4   187   72000   07/06/2006   20:00:00
Eph Epoch:    1382( 358)  417600   Thu-4   187   72000   07/06/2006   20:00:00
Transmit Week:1382
Fit interval flag :  0

          SUBFRAME OVERHEAD

               SOW    DOW:HH:MM:SS     IOD    ALERT   A-S
SF1 HOW:    411426  Thu-4:18:17:06   0x17D      0      on
SF2 HOW:    411432  Thu-4:18:17:12    0x7D      0      on
SF3 HOW:    411438  Thu-4:18:17:18    0x7D      0      on

           CLOCK
. . .

\end{verbatim}
\end{\outputsize}
\subsection{Notes}
Both files can either be a RINEX or a FIC file.

%\end{document}

