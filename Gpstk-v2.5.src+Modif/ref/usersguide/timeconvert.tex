%\documentclass{article}
%\usepackage{fancyvrb}
%\usepackage{perltex}
%\usepackage{xcolor}
%\usepackage{listings}
%\usepackage{longtable}
%\usepackage{multirow}
%\RecustomVerbatimEnvironment{Verbatim}{Verbatim}{frame=single}
\definecolor{console}{rgb}{0.95,0.95,0.95}
\lstset{basicstyle=\ttfamily, columns=flexible, backgroundcolor=\color{console}}

\newcommand{\outputsize}{footnotesize}
\newcommand{\application}[1]{\emph{#1}}
\newcommand{\setconsole}{\lstset{basicstyle=\ttfamily, columns=flexible, backgroundcolor=\color{console}}}
\newcommand{\setfileio}{\lstset{basicstyle=\ttfamily, columns=flexible, backgroundcolor=\color{console}}}

\perlnewcommand{\getuse}[1]
{
        my $command = $_[0];
        $command = $command." > temp 2>&1 |head -n 15";
        system("$command");

	my $counter = 0;
	my $done = 0;
	my $output = "";
        open(input,"temp");

        while(my $line = <input>)
	{
		if($done == 0)
		{
			$output = $output.$line if $counter < 15;
			$output = $output." . . .\n" if $counter >= 15;
			$done = 1 if $counter >= 15;
		}

		$counter = $counter + 1;
	}

        close(input);

        return  "\\begin{\\outputsize}\n" . "\\begin{lstlisting}\n" .
                "> ".$_[0]."\n\n" . $output .
                "\\end{lstlisting}\n" . "\\end{\\outputsize}\n";
}

\perlnewcommand{\getrevision}[1]
{
	my $revision = $_[0];
	$revision =~ m/\$LastChangedRevision: ([^\$]*)/;
	return $1;
}

\perlnewcommand{\entry}[4]
{
	my $output = "$_[0] \& $_[1] \& \\multirow{$_[3]}{2.5in}{$_[2]} \\\\";

	my $cnt = 0;
	while($cnt < ($_[3]-1))
	{
		$output = $output." \& \& \\\\ ";
		$cnt = $cnt + 1;
	}
	
	return $output;
}


%\begin{document}

\index{timeconvert!application writeup}
\section{\emph{timeconvert}}
\subsection{Overview}
This application allows the user to convert between time formats associated with 
GPS. Time formats include: civilian time, Julian day of year and year, GPS week 
and seconds of week, Z counts, and Modified Julian Date (MJD).

\subsection{Usage}
\subsubsection{\emph{timeconvert}}
\begin{\outputsize}
\begin{longtable}{lll}
%\multicolumn{3}{c}{\application{timeconvert}}\\
\multicolumn{3}{l}{\textbf{Optional Arguments}} \\
\entry{Short Arg.}{Long Arg.}{Description}{1}
\entry{-d}{--debug}{Increase debug level.}{1}
\entry{-v}{--verbose}{Increase verbosity.}{1}
\entry{-h}{--help}{Print help usage.}{1}
\entry{-A}{--ansi=TIME}{``ANSI-Second".}{1}
\entry{-c}{--civil=TIME}{``Month(numeric) DayOfMonth Year Hour:Minute:Second}{2}
\entry{-R}{--rinex-file=TIME}{``Year(2-digit) Month(numeric) DayOfMonth Hour Minute Second".}{2}
\entry{-o}{--ews=TIME}{``GPSEpoch 10bitGPSweek SecondOfWeek".}{1}
\entry{-f}{--ws=TIME}{``FullGPSWeek SecondOfWeek".}{1}
\entry{-w}{--wz=TIME}{``FullGPSWeek Zcount".}{1}
\entry{}{--z29=TIME}{``29bitZcount".}{1}
\entry{-Z}{--z32=TIME}{``32bitZcount".}{1}
\entry{-j}{--julian=TIME}{``JulianDate".}{1}
\entry{-m}{--mjd=TIME}{``ModifiedJulianDate".}{1}
\entry{-u}{--unixtime=TIME}{``UnixSeconds UnixMicroseconds".}{1}
\entry{-y}{--doy=TIME}{``Year DayOfYear SecondsOfDay".}{1}
\entry{}{--input-format=ARG}{Time format to use on input.}{1}
\entry{}{--input-time=ARG}{Time to be parsed by "input-format" option.}{1}
\entry{-F}{--format=ARG}{Time format to use on output.}{1}
\entry{-a}{--add-offset=NUM}{Add NUM seconds to specified time.}{1}
\entry{-s}{--sub-offset=NUM}{Subtract NUM seconds from specified time.}{1}
\end{longtable}
\end{\outputsize}

\subsection{Examples}
\begin{\outputsize}
\subsubsection{Convert RINEX file time.}
\begin{verbatim}
> timeconvert -R "05 06 1985 13:50:02"

        Month/Day/Year H:M:S            11/06/2010 13:00:00
        Modified Julian Date            55506.541666667
        GPSweek DayOfWeek SecOfWeek     584 6  565200.000000
        FullGPSweek Zcount              1608 376800
        Year DayOfYear SecondOfDay      2010 310  46800.000000
        Unix: Second Microsecond        1289048400      0
        Zcount: 29-bit (32-bit)         306560992 (843431904)
\end{verbatim}

\subsubsection{Convert ews time.}
\begin{verbatim}
timeconvert -o "01 1379 500"

Month/Day/Year                    1/25/2026
Hour:Min:Sec                      00:08:20
Modified Julian Date              61065.005787037
GPSweek DayOfWeek SecOfWeek       355 0 500.000000
FullGPSweek Zcount                2403 333
Year DayOfYear SecondOfDay        2026 25 500.000000
Unix_sec Unix_usec                1769299700 0
Zcount: 29-bit (32-bit)           186122573 (1259864397)
\end{verbatim}
\end{\outputsize}
\subsection{Notes}
If no arguments are given it will convert the current time to all formats. When inputting time values, include quotation marks.

%\end{document}
