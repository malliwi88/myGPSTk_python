%\documentclass{article}
%\usepackage{perltex}
%\usepackage{fancyvrb}
%\usepackage{xcolor}
%\usepackage{listings}
%\usepackage{longtable}
%\usepackage{multirow}
%\RecustomVerbatimEnvironment{Verbatim}{Verbatim}{frame=single}
\definecolor{console}{rgb}{0.95,0.95,0.95}
\lstset{basicstyle=\ttfamily, columns=flexible, backgroundcolor=\color{console}}

\newcommand{\outputsize}{footnotesize}
\newcommand{\application}[1]{\emph{#1}}
\newcommand{\setconsole}{\lstset{basicstyle=\ttfamily, columns=flexible, backgroundcolor=\color{console}}}
\newcommand{\setfileio}{\lstset{basicstyle=\ttfamily, columns=flexible, backgroundcolor=\color{console}}}

\perlnewcommand{\getuse}[1]
{
        my $command = $_[0];
        $command = $command." > temp 2>&1 |head -n 15";
        system("$command");

	my $counter = 0;
	my $done = 0;
	my $output = "";
        open(input,"temp");

        while(my $line = <input>)
	{
		if($done == 0)
		{
			$output = $output.$line if $counter < 15;
			$output = $output." . . .\n" if $counter >= 15;
			$done = 1 if $counter >= 15;
		}

		$counter = $counter + 1;
	}

        close(input);

        return  "\\begin{\\outputsize}\n" . "\\begin{lstlisting}\n" .
                "> ".$_[0]."\n\n" . $output .
                "\\end{lstlisting}\n" . "\\end{\\outputsize}\n";
}

\perlnewcommand{\getrevision}[1]
{
	my $revision = $_[0];
	$revision =~ m/\$LastChangedRevision: ([^\$]*)/;
	return $1;
}

\perlnewcommand{\entry}[4]
{
	my $output = "$_[0] \& $_[1] \& \\multirow{$_[3]}{2.5in}{$_[2]} \\\\";

	my $cnt = 0;
	while($cnt < ($_[3]-1))
	{
		$output = $output." \& \& \\\\ ";
		$cnt = $cnt + 1;
	}
	
	return $output;
}


%\begin{document}

\index{calgps!application writeup}
\section{\emph{calgps}}
\subsection{Overview}
This application generates a dual GPS and Julian calendar to either stdout or to a graphics file. The arguments and 
format are inspired by the UNIX `cal' utility. With no arguments, the current 
argument is printed. The last and next month can also be printed. Also, the 
current or any given year can be printed.
\subsection{Usage}
\subsubsection{\emph{calgps}}
\begin{\outputsize}
\begin{longtable}{lll}
%\multicolumn{3}{c}{\application{calgps}} \\
\multicolumn{3}{l}{\textbf{Optional Arguments}} \\
Short Arg. & Long Arg. & Description \\
-h & --help & Generates help output. \\ 
-3 & --three-months & Prints a GPS calendar for the previous, current, and next month. \\
-y & --year & Prints a GPS calendar for the entire current year. \\
-Y & --specific-year=NUM & Prints a GPS calendar for the entire specified year. \\
-p & --postscript=ARG  & Generates a postscript file. \\
-s & --svg=ARG & Generates an SVG file. \\
-e & --eps=ARG & Generates an encapsulated postscript file. \\
-v & --view & Try to launch an appropriate viewer for the file. \\
-n & --no-blurb & Suppress GPSTk reference in graphic output. \\

\end{longtable}
\end{\outputsize}

\subsection{Examples}
\begin{\outputsize}
\begin{verbatim}
> calgps -3

                       Jun 2011
1638                        1-152  2-153  3-154  4-155 
1639   5-156  6-157  7-158  8-159  9-160 10-161 11-162 
1640  12-163 13-164 14-165 15-166 16-167 17-168 18-169 
1641  19-170 20-171 21-172 22-173 23-174 24-175 25-176 
1642  26-177 27-178 28-179 29-180 30-181               

                       Jul 2011
1642                                      1-182  2-183 
1643   3-184  4-185  5-186  6-187  7-188  8-189  9-190 
1644  10-191 11-192 12-193 13-194 14-195 15-196 16-197 
1645  17-198 18-199 19-200 20-201 21-202 22-203 23-204 
1646  24-205 25-206 26-207 27-208 28-209 29-210 30-211 
1647  31-212                                           

. . .
\end{verbatim}
\end{\outputsize}
\subsection{Notes}
If multiple options are given only the first is considered.

%\end{document}

