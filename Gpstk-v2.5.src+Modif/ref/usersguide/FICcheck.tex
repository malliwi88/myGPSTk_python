%\documentclass{article}
%\usepackage{fancyvrb}
%\usepackage{perltex}
%\usepackage{xcolor}
%\usepackage{listings}
%\usepackage{longtable}
%\usepackage{multirow}
%\RecustomVerbatimEnvironment{Verbatim}{Verbatim}{frame=single}
\definecolor{console}{rgb}{0.95,0.95,0.95}
\lstset{basicstyle=\ttfamily, columns=flexible, backgroundcolor=\color{console}}

\newcommand{\outputsize}{footnotesize}
\newcommand{\application}[1]{\emph{#1}}
\newcommand{\setconsole}{\lstset{basicstyle=\ttfamily, columns=flexible, backgroundcolor=\color{console}}}
\newcommand{\setfileio}{\lstset{basicstyle=\ttfamily, columns=flexible, backgroundcolor=\color{console}}}

\perlnewcommand{\getuse}[1]
{
        my $command = $_[0];
        $command = $command." > temp 2>&1 |head -n 15";
        system("$command");

	my $counter = 0;
	my $done = 0;
	my $output = "";
        open(input,"temp");

        while(my $line = <input>)
	{
		if($done == 0)
		{
			$output = $output.$line if $counter < 15;
			$output = $output." . . .\n" if $counter >= 15;
			$done = 1 if $counter >= 15;
		}

		$counter = $counter + 1;
	}

        close(input);

        return  "\\begin{\\outputsize}\n" . "\\begin{lstlisting}\n" .
                "> ".$_[0]."\n\n" . $output .
                "\\end{lstlisting}\n" . "\\end{\\outputsize}\n";
}

\perlnewcommand{\getrevision}[1]
{
	my $revision = $_[0];
	$revision =~ m/\$LastChangedRevision: ([^\$]*)/;
	return $1;
}

\perlnewcommand{\entry}[4]
{
	my $output = "$_[0] \& $_[1] \& \\multirow{$_[3]}{2.5in}{$_[2]} \\\\";

	my $cnt = 0;
	while($cnt < ($_[3]-1))
	{
		$output = $output." \& \& \\\\ ";
		$cnt = $cnt + 1;
	}
	
	return $output;
}


%\begin{document}

\index{ficacheck!application writeup}
\index{ficcheck!application writeup}

\section{\emph{ficacheck ficcheck}}
\subsection{Overview}
These applications read input ASCII or binary FIC and check them for errors. \application{ficcheck} checks binary files and \application{ficacheck} checks ASCII files.

\subsection{Usage}
\subsubsection{\emph{ficacheck ficcheck}}

\begin{\outputsize}

\begin{longtable}{lll}
\multicolumn{3}{l}{\textbf{Optional Arguments}} \\
\entry{Short Arg.}{Long Arg.}{Description}{1}
\entry{-d}{--debug}{Increase debug level.}{1}
\entry{-v}{--verbose}{Increase verbosity.}{1}
\entry{-h}{--help}{Print help usage.}{1}
\entry{-t}{--time=TIME}{Time of first record to count (default BOT).}{1}
\entry{-e}{--end-time=TIME}{End of time range to compare (default EOT).}{1}
\end{longtable}
\begin{verbatim}
ficacheck usage: ficacheck [options] <FICA file>
ficcheck usage:  ficcheck [options] <FIC file>
\end{verbatim}
\end{\outputsize}

\subsection{Examples}
\begin{\outputsize}
\begin{verbatim}
>ficcheck fic06.187

Checking fic06.187
Read 252 records.


> ficacheck brokenfica

Checking brokenfica
text 0:Bad block header, record=2 location=484
text 1:blkHdr=[    ]
text 2:In record 2
text 3:In file brokenfica
text 4:Near file line 10
location 0:src/FICData.cpp:963
location 1:src/FFStream.cpp:159
location 2:src/FFStream.hpp:208
location 3:src/FFStream.hpp:208
\end{verbatim}
\end{\outputsize}
\subsection{Notes}
Only the first error in each file is reported. The entire file is always checked regardless of time options.

%\end{document}
