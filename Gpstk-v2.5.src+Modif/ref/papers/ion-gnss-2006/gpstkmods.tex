
\section*{Project Description}

As a project, the GPSTk has changed signficantly since it was
presented at the ION-GNSS-2005. New processes have been implemented to
support both development of code and use of the
applications. Development support includes a shared, online repository
hosted by Source Forge\cite{sourceforge}, and a new testing
framework. Users are now supported with an online, collaborative
project site. In addition, a user manual for the GPSTk applications
has been initiated and is publicly available.


--------------------------------------

\subsection*{Data Collection and Conversion}

There are applications that convert observations between RINEX and
other formats. Some applications such as \gpstkapplication{novaRinex}
and \gpstkapplication{rtAshtech} parse message specific to a
particular family of receivers. Other applications such as
\gpstkapplication{RinexDump} and \gpstkapplication{navdmp} that
reformat RINEX into human- or machine-readable formats.

There are two sets of conversion applications that support formats
derived by ARL:UT.  The first set supports a format known as
Floating-Integer-Character (FIC). FIC was first proposed in the 1980's
as a format to record a broad range of measurments from a GPS receivers or
a reference station. \cite{rinex1format}, \cite{ficproposal}. The full
FIC format supports the recording of pseudorange observations and
navigation messages, as well as atomic clock models. In a sense FIC is
a precursor to BINEX. In practice, FIC is only used to store the
navigation message. Therefore, in the GPSTk, FIC functionality is
restricted to that pertaining to the navigation message. The
second set of applications that support formats derived by ARL:UT use
the Measurement Data Port (MDP) protocol. The MDP format and its
associated tools will be discussed in the following ``New
Applications'' section.
